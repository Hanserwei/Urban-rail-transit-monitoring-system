\begin{cnabstract}
本设计的重点是开发一个综合监控系统(ISCS),用于城市轨道交通,特别是天津地铁6号线。ISCS系统是一个分层分布式的计算机集成系统,主要用于供电系统设备的远程监控和遥控。该系统在轨道交通系统的安全运行中起着关键作用,实现了对多种设备的监控,包括信息采集、数据分析处理和事故报警功能。

设计过程中首先对综合监控系统的发展历程进行了回顾,以了解其从最初的简单系统到现代复杂系统的演变。基于对国内相关文献的研究和需求分析,为天津地铁6号线工程制定了一个切实可行的设计方案。该方案深入考虑了轨道交通监控系统的各种组成需求和功能,以确保系统能够满足实际运行中的各种场景和要求。
总体方案的设计考虑了系统的灵活性、可维护性和未来升级的可能性,确保系统能够适应不断变化的需求。使用Visio软件绘制了ISCS系统的直观而清晰的结构示意图,为团队提供了一个全面了解系统组成和关系的视觉参考。在FAS(火灾自动报警系统)子系统的设计中,详细划分了各个功能模块,并考虑了四个遥控点表的配置,以确保系统在监控和遥控方面的优异表现。

通过这次设计项目,不仅完成了天津地铁6号线综合监控系统的总体设计方案,还提供了FAS子系统的详细设计方案,以满足需求书中的设备要求和功能要求
\end{cnabstract}
\vspace{1em}\par

\cnkeywords{FAS系统;城市轨道交通;综合监控系统;四遥点表}
