\chapter{设计背景概述}
\section{设计背景}
城市轨道交通是指在固定轨道上行驶的交通工具,包括地铁、轻轨、有轨电车等。中国的城市轨道交通发展经历了不同的历史时期,主要包括早期有轨电车交通(20世纪初至20世纪50年代)和现代城市轨道交通(20世纪60年代中期至今)两个时期。

自1965年北京地铁1号线开工建设以来,中国的城市轨道交通得到了长足的发展。截至2021年底,我国已有70个城市开通了地铁,总运营里程超过8000公里。此外,我国在高铁、城际铁路等领域也取得了显著的进展。国务院于2021年12月印发了《“十四五”现代综合交通运输体系发展规划》,提出了建设交通强国、构建现代综合交通运输体系的目标。

综合监控系统的发展可以追溯到20世纪80年代。随着计算机技术和通信技术的不断进步,综合监控系统得到了长足的发展。目前,综合监控系统广泛应用于各个领域,包括能源、交通、环保、安防等。

在城市轨道交通领域,综合监控系统(ISCS)是一个高度集成的综合自动化监控系统,用于集中监控和管理地铁的强弱电设备。ISCS通过集成多个主要电力系统,形成统一的监控层硬件和软件平台,实现对设备的监控、管理和数据关联监视。该系统是一个分层分布式计算机集成系统,可以监控各种设备、进行数据采集、数据分析处理以及事故报警等控制功能。

综合监控系统在城市轨道交通中起着重要作用,它通过集成多个主要强弱电系统,形成统一的监控层硬件和软件平台,实现对地铁设备的集中监控和管理,同时实现列车运行情况和客流统计数据的关联监视。这有助于提高地铁的安全性和运行效率。

综合监控系统的发展是一个不断完善的过程,随着技术的不断创新,将为城市轨道交通的未来提供更多可能性。

\section{综合监控系统发展史}
二十年前,计算机与自动化控制技术的发展与应用,极大地提高了轨道交通的自动化监测与管理水平。这直接促使了许多分散的电机控制系统,如BAS(建筑自动化系统,用于环境与设备监测)、FAS(火警报警系统)和SCADA(监视与数据采集控制)系统的形成。这些独立的电机系统各自运作,但在一定程度上进行了基本的信息交流,比如火警信息通过干接点或总线系统告知其他系统,触发其进入灾难模式。

大约十五年前,中国开始探索集成系统的方向。2002年,北京地铁13号线采用了统一的IT结构和软件,初步集成了BAS、FAS和SCADA三个系统。尽管规模相对较小,仅有一个地下站,但这标志着成功的尝试。深圳地铁1号线的第一阶段工程也采用了类似的模式进行集成,并互联了各种系统信息。这些早期的尝试可以看作是对综合监控系统的初步探索。

广州地铁3号线的主控系统的设计与实施对综合监控系统的形成起到了关键作用。该系统集成并互联了几乎所有电机系统。随着调度和运营需求不断提升,越来越需要更加有效的前台数据处理(调度功能)和后台数据应用(数据分析和维护支持)。综合监控系统不再关注子系统是集成还是互联的问题,而是旨在提高系统应用功能和效率。

在这个过程中,除了完善联动功能,综合监控系统还在以下几个方面进行了重要的努力:
\begin{enumerate}
	\item 集成多媒体系统,提供更一致的调度服务。
	\item 集成车载信息和视频,有效监视车辆状态。
	\item 应用电力并行运行控制卡,提高维护过程中的电源供应效率。
	\item 建立分级处理调度-维护数据和设备管理系统,支持运营和维护管理。
\end{enumerate}
1. 集成多媒体系统,提供更一致的调度服务。
2. 集成车载信息和视频,有效监视车辆状态。
3. 应用电力并行运行控制卡,提高维护过程中的电源供应效率。
4. 建立分级处理调度-维护数据和设备管理系统,支持运营和维护管理。

这些努力取得了良好的成果。除了信息技术的发展和运营需求的推动外,主要系统集成商对综合监控系统建设需求的认知增强以及软件平台的本土化在这一时期发挥了关键作用。

这一模式已经持续了大约八年时间,使综合监控系统成为一个成熟和高效的系统。其两级管理和三级控制系统至今仍在使用中。

\section{设计原则和标准}
\subsection{主要设计原则}
\begin{itemize}
	\item 城市轨道交通电力监控系统的主要设计原则:
	\begin{enumerate}
		\item 电力监控系统由中央监控系统、变电所综合自动化系统、远动通道组成。
		\item 中央监控系统采用1:N结构计算机型电力集中监控装置。
		\item 采用问答式通讯规约,通道结构为点对点方式。
		\item 通道采用地铁综合通信光纤网中的专用通道,并设置主用和备用通道,主备通道能实现手动及自动切换。
		\item 中央监控系统具备与其他系统的通信接口能力。
		\item 系统设备选型立足于国产化。
	\end{enumerate}
\end{itemize}
\subsection{主要设计标准}
\begin{itemize}
	\item 城市轨道交通电力监控系统参考以下设计标准:
	\begin{enumerate}
		\item 《远动系统和设备》(IEC870-88)
		\item 《地区电网数据采集与监控系统通用技术条件》(GB/T13730-92)
		\item 《远动终端通用技术条件》(GB/T13729-92)
		\item 《计算机场地技术要求》(GB/2887-89)
		\item 《地下铁道设计规范》(GB50157-92)
		\item 《地铁直流牵引供电系统设计规范》(GB/T10411-89)
		\item 《铁路电力牵引供电远动系统技术规范》(TB10117-98)
	\end{enumerate}
\end{itemize}

\section{集成原则与系统构成}
根据应用需求,将硬件平台、网络设备、系统软件、工具软件以及相关的应用软件整合成一套性价比出色的计算机系统。
\subsection{集成原则}
地铁的基本运营状态包括正常运营状态、夜间停止运营状态和紧急运营状态。地铁运营服务就是在这三种状态下,保证人员和设备的安全,提供人性化服务,从而提高地铁运营管理效率。

现代化的地铁运营管理要求自动化系统能提供一个可实现信息互通和资源共享的平台。综合监控系统采用通用性好、符合国际标准或行业标准的、高可靠性的网络交换机、服务器和工控机等网络和计算机产品来构建统一硬件集成平台。综合监控系统采用模块式、类似积木结构的多层软件开发平台定制应用软件,以集成和互联的方式与各接入系统实现信息交换。综合监控系统以通用开放的硬件接口及软件通信协议为基础,最终实现对各相关机电设备的集中监控功能和各系统之间的信息互通、信息共享和协调互动功能。

综合监控系统分为两种方式:集成方式和互联方式。集成方式是指被集成子系统的中央车站级上位机的监控功能皆由综合监控系统实现。脱离了综合监控系统,各集成系统原有的上位机监控功能将难以实现。互联方式是指定互联接入系统,其自身是一个独立系统,可脱离综合监控系统单独工作。互联系统只是将一些运营所需的信息上传至综合监控系统,从而实现各机电系统之间的信息互通和协调互动功能。

综合监控系统主要包括两大部分:对机电设备的实时集中监控功能和各系统之间协调联动功能。通过综合监控系统,可实现对电力设备、火灾报警信息及其设备、车站环控设备、区间环控设备、环境参数、屏蔽门设备、防淹门设备、电扶梯设备、照明设备、门禁设备、自动售检票设备、广播和闭路电视设备、乘客信息显示系统的播出信息和时钟信息等进行实时集中监视和控制的基本功能。

另一方面,通过综合监控系统,还可实现晚间非运营情况下、日间正常运营情况下、紧急突发情况下和重要设备故障情况下各相关系统设备之间协调互动等高级功能。综合监控系统的集成平台示意如图1-2所示。

\subsection{系统的基本构成}
1. 城市轨道交通综合监控系统硬件概要

第一层:中央级综合监控系统

此层包含备用实时服务器、冗余历史服务器、外部磁盘存储、磁带存储设备、中央前端处理器(FEP)、多种调度员工作站(例如电调、环调、行调、值班调度和值班主任助理)、网络管理服务器、网络管理工作站、软件测试平台服务器、事件记录打印机、报表打印机、彩色图形打印机、备用带路由功能网络交换机、大型屏幕系统(OPS)、不间断电源(UPS)等。

第二层:车站级综合监控系统
。
这一层包括备用车站级服务器、外部磁盘存储、值班站长工作站、事件记录打印机、报表打印机、备用带路由功能网络交换机、车站前端处理器(FEP)、综合备份盘(IBP)和UPS等。

2. 城市轨道交通综合监控系统软件架构

第一层:数据接口层

这一层专用于数据采集和协议转换,主要包括综合监控系统前端处理器(FEP)。它通过FEP实现与相关系统的数据通信,包括数据采集、协议转换和数据隔离等功能。

第二层:数据处理层

这一层专注于数据处理,主要包括车站服务器和中央服务器,通过实时数据库和关系数据库提供ISCS应用功能。

第三层:人机接口层

这一层专门处理人机接口,主要包括操作员工作站。通过网络从车站和中央服务器获取数据,并在工作站上显示人机界面。

3. 城市轨道交通综合监控系统集成分类

一、信息综合管理系统

信息综合管理系统是在OCC建立一个局域网,将已建好的城轨各分立自动化系统的必要信息综合在一起,实现信息共享,但只监督不控制。

二、顶层信息集成的综合监控系统

顶层信息集成的综合监控系统是在OCC和车站的监控层将部分子系统集成和互联起来构成综合监控系统。

三、深度集成的综合监控系统

深度集成的综合监控系统采用同一软件平台将被集成的子系统完全集成在一起。被集成子系统的中央层、车站监控层和控制层被集成在综合监控平台上,它们的功能都由综合监控软件来实现。深度集成的综合监控系统以数个被集成子系统的集成平台为基础,再将被互联子系统接入,构建一个功能强大、体系结构完整的综合监控系统。

4. 城市轨道交通综合监控系统构成原则

1)综合监控系统应围绕行车和行车指挥、防灾和安全、乘客服务等开展设计,以进一步提高运营行车管理的水平。

2)综合监控系统面向的对象为控制中心的各中央调度员(行调、电调、环调、值班调度和值班主任助理)、车站控制室和停车场消防控制室的值班人员和车辆段维修中心的系统维护人员等。综合监控系统应满足以上这些岗位的功能要求。

3)综合监控系统的故障告警功能,分别在控制中心及停车场维修中心实现,在控制中心综合监控系统应能采集相关集成系统的重要设备故障的汇总信息,以方便中央维调人员的维护管理工作;另外在停车场维修中心应能采集相关集成系统的重要设备故障信息,并具备对所采集信息进行汇总统计的功能,从而方便停车场维修人员进行日常的系统设备的维护工作。

4)当出现异常情况由正常运行模式转为灾害运行模式时,综合监控系统应能迅速转变为应急模式,为防灾、救援和事故处理指挥提供方便。

5)地铁自动化系统应由上位监控层、中间控制层和末端设备层三层构成;综合监控系统属于上位监控层,是由控制中心、车站综合监控系统的交换机、服务器、工作站和前置处理器(FEP)等设备组成;中间控制层和末端设备层由相关接入系统和现场设备组成。

6)控制中心与车站上位监控层的计算机设备通过工业级骨干传输网络连接。上位监控层与中间控制层设备主要通过符合国际或行业标准的通用开放式的智能通信接口形式进行连接。中间控制层与末端设备层主要通过通用开放式的工业控制网络、现场总线和硬线等接口形式进行连接。

7)综合监控系统应根据各集成系统的实际需求向相关集成系统开放全线骨干网络资源,为集成系统具有逻辑上独立的全线网络传输通道,并保证综合监控系统网络安全。

8)综合监控系统应能实时反映各监控对象的工作状态,综合监控系统应具备对监控对象的进行模式控制、程序控制、时间表控制和点动控制等控制功能。

9)地铁弱电系统的安全联锁控制功能主要在中间控制层实现。控制层设备应具备相对独立的工作能力,即控制层设备脱离中央或车站信息管理层时,仍能独立运行,满足紧急情况下运营的应急需求。

10)综合监控系统应采用模块化设计,易于扩展。综合监控系统不仅应满足某地铁线运营管理的需求,还应考虑线路扩展的需求,同时还应为其他线路的接入和更高一级管理系统的连接预留一定的条件。

11)综合监控系统换乘站方案考虑两条相关线路的建设时间的先后次序等因素,采用分别设置综合监控系统及相关接入系统的方案。

12)综合监控系统应采用高可靠的产品,保证能全天候不间断地运行。

\section{设计主要内容}

本设计为天津地铁6号线综合监控系统以及FAS子系统的方案设计,内容包括如下:

第1章内容为综合监控系统的发展历程以及集成方式与结构,并对综合监控系统的三层次系统构成进行简要说明;

第2章内容为综合监控系统总体设计,包括综合监控系统功能分析,集成互联方案,设备具体构成,中心、典型车站、车辆段、网络的设计方案;

第3章内容为综合监控系统体系下的FAS子系统设计的简介及分析,包括FAS系统的应用分析,功能分析,与BAS的接口稳定性分析以及消防联动功能的应用分析。

第4章内容在FAS子系统设计的简介及分析的基础上,对FAS系统的触发器件,控制装置,传感器设备进行介绍与选型。最后对FAS接口进行分析。
