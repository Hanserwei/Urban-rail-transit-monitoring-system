\zihao{-4}

\vspace*{0.5cm}
\begin{center}{\hei \zihao{-2} \textbf{课程设计任务书}}\end{center}
\vspace{1cm}

\hspace{-0.85cm}班\ \ 级\underline{\makebox[3.91cm]{智控2020级4班}}
学生姓名\underline{\makebox[3.91cm]{伍维}}
学\ \ 号\underline{\makebox[3.91cm]{2020117568}}
\\发题日期:2023年9月18日\hfill 完成日期:2023年11月17日
\\
\\题\ \ 目\underline{\makebox[14.55cm]{天津地铁6号线综合监控系统总体及FAS系统设计}}
\\1、本设计(论文)的目的、意义

\uline{综合供变电工程、继电保护、城轨监控技术等专业课程的学习,综合应用所学课程知识,了解轨道交通综合监控系统的构成,熟悉有关“规程”和“设计手册”的使用方法。初步掌握城市轨道交通电力监控系统设计步骤和方法。}
\\2、学生应完成的任务\\
\noindent
\uline{\\
	1. 天津地铁六号线综合监控系统招标书、车站平面布置图、供电设备布置图\\
	2 .城市轨道交通相关设计规范、标准\\
	3 .设计该线路综合监控系统总体技术方案,并对FAS系统进行详细设计\\
}
\\3、课程设计主要任务:

\uline{\\
	1. 根据招标书技术条件,给出综合监控系统集成总体技术方案,总体技术方案应包括主站技术方案、子站技术方案、网络技术方案;\\
	2 .对1中的技术方案给出主要设备配置、参数要求、功能描述;\\
	3 .对招标书中要求的FAS系统,选择典型车站进行详细设计,给出系统详细构成、主要设备选型及功能。\\
	4 .根据FAS设计方案,按照IEC60870-5-101制定遥信、遥测、遥控、遥调点表。
}

\newpage

\hspace{-0.85cm}4、本设计(论文)各部分内容及时间分配:(共\underline{\makebox[1cm]{8}}周)
\\第一部分\uline{\quad 根据招标书进行功能分析 \quad}\hfill(1周)
\\第二部分\uline{\quad 综合监控集成方案设计 \quad}\hfill(2周)
\\第三部分\uline{\quad 子系统详细设计 \quad}\hfill(2周)
\\第四部分\uline{\quad 点表制定 \quad}\hfill(1周)
\\第五部分\uline{\quad 论文写作 \quad}\hfill(2周)
\vspace{16cm}
\\指导教师:\hspace{4cm}年\hspace{1cm}月\hspace{1cm}日
% \\审\ \ 批\ \ 人:\hspace{4cm}年\hspace{1cm}月\hspace{1cm}日